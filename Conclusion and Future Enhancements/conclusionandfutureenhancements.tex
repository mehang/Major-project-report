\newpage

\section{CONCLUSION}
\subsection{Conclusion}
Automatic Music classification is an interesting problem that has gained prominence with the rise of the Internet and the subsequent growth in the amount of music available to the general populace.\\
\\
Any type of classification of music is difficult simply because the classifications themselves don’t have a clear definition. 
Still, we can work with fuzzy boundaries between these classes to get good enough results with Music Classification Systems.\\
\\
These systems broadly consist of two steps: Feature Extraction and Classification. However, they can be created by combining a variety of techniques and components.\\
\\
So, in this project, many such components and approaches were studied such as: types and combinations of features for 
proper representation of songs, feature integration approaches, classifier types, and their parameters, etc.\\
\\
All these studies were done in order to tackle two related but distinct problems: 
\begin{itemize}
        \item In Automatic Music Genre Classification (AMGC), good performances were achieved with both of the classifiers employed: the final SVM model got 83 per cent accuracy while the ANN model got 88 per cent accuracy for five genres. These results are comparable with the state-of-the-art results, especially involving the same dataset. 
        \item In Music Mood Classification however, the good results couldn’t be replicated. 
                The result along both axes of the music mood model used (arousal and valence) were underwhelming. Around 73 per cent accuracy was 
                achieved using ANN for the binary low/high arousal classification. SVM did even worse with around 70 per cent accuracy. 
                For low/high valence classification, both of the classifiers settled on 67 per cent accuracy.  
\end{itemize}

\subsection{Future Enhancements}
There are many ways in which the project could be expanded. 
\begin{itemize}
        \item Adding many more genres would be a natural enhancement as the five genres used here don’t even come close to encompassing all the variety of musical genres. 
        \item The mood model could also be made a continuum in the two-dimensional space instead of hard binary classifiers. 
        \item More features could be added to the feature set and studies conducted as to which combinations of features work best in which problem domain. 
        \item Smart Playlist Generation System could also be built using these types of Automatic Music Classification Systems. 
        \item The training set used to build the classifier could be increased in both size and variety to make the models generated more robust. This could also be a topic of study.
\end{itemize}





