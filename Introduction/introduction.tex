\newpage
\section{INTRODUCTION}
\pagenumbering{arabic}

\subsection{Background}

\par Due to the rapid development in the music industry, there has been an increasing amount of work in the area of automatic genre classification
of music in audio format. One of the prime area focused on by MIR is the Automatic Classification of such music based on signal analysis. Such systems
can be used as a way to evaluate features describing music content as well as a way to structure large collections of music.\\
\par The presence of huge amount of data in music databases can be a perfectly applicable area. Music is complex in nature not only due to its origination
but also due to the evolution of music in the technological era. So it requires specialized representations, abstraction and processing techniques for effective
analysis, evaluation and classification that are fundamentally different form those used for other mediums and tasks.\\

\subsection{Motivation}
The presence of numerous genre is a source of confusion and more often than not people are overwhelmed with the sheer vastness of music available. As
such, the primary motivation is to make it easier for people to classify music (based on genre and/or mood) so that they can find songs suited to their
own tastes.\\ 
\par It can also lay the foundation for figuring out ways to represent similarity between two musical pieces and in the making of a
good recommendation system.\\ 

\subsection{Problem Statement}
Though the project is focused on music classification based genre and mood, we focused only on consolidating the genre part for this mid-term
milestone. The mood based classification has been left for the next semester. Although the number of genres present worldwide is huge, we have
restricted ourselves to five major genres which are: Rock, Pop, Classical, Jazz and Hiphop.\\ 
\par The choice of these genres is based on their being sufficiently distinguishable from each other. Choosing some genre that’s very unique and
abnormal might have made them more distinguishable and easier to classify but it would have been harder to find quality data/works for those genre.
So, we chose these genre with availability of musical pieces in mind too.

\subsection{Objectives}
\begin{enumerate}[i.)]
        \item To study and implement different preprocessing steps involved in extracting features from audio data.
        \item To implement suitable classification algorithm for various features of the song
        \item To cross validate the result and analyze the efficiency of the algorithms used.
\end{enumerate}

\subsection{Significance and Scope}

\subsubsection{Significance}
Music classification based on genre is the current topic on hype in the musical sector. Music Information Retrieval(MIR) has put on a huge interest
in this sector. The availability of huge music databases can be well organized by use of such classification system. 

\subsubsection{Scope of work}
\begin{enumerate}[i)]
        \item The project will work on classifying music based on genre and mood. More specifcally, the classifcation will be done on western music only as the data
                is more easily available and lots of works have been done in the past for it. Also, only five genres will be used for genre classifcation:
                \begin{itemize}
                        \item Rock
                        \item Pop
                        \item Classical
                        \item Jazz
                        \item Hiphop
                \end{itemize}

        \item The mood based classifcation will use the Thayer model, a two dimensional model based on Energy and Stress:

                \begin{itemize}
                        \item High Energy, High Stress = Anxious/Frantic

                        \item High Energy, Low Stress = Exuberance

                        \item Low Energy, High Stress = Depression

                        \item Low Energy, Low Stress = Contentment
                \end{itemize}

        \item Also, it is entirely possible for a song or a piece of music to fall into multiple genre or moods. The characteristics that defne the genre
                and the mood may change within the song itself with one part showing seeming to belong to one class while other parts may seem to belong to
                an entirely diferent class. The project will not cover such issues. In other words, multiple-tagging will not be done.

        \item The classification will work only on mp3 files. No other fle format will be supported in the initial release/version.

\end{enumerate}
